\documentclass[12pt]{article}
\usepackage[utf8]{inputenc}
\usepackage[portuguese]{babel}
\usepackage{graphicx}
\usepackage{authblk}
\usepackage{hyperref}
\usepackage{biblatex}
\usepackage{amsthm}
\usepackage{amsmath}
\usepackage{setspace}
\usepackage[T1]{fontenc}
\usepackage[bottom=2cm,top=3cm,left=3cm,right=2cm]{geometry}

\hyphenation{qua-drá-ti-co}

\onehalfspace

\theoremstyle{definition}
\newtheorem{definition}{Definition}[section]
\newtheorem{theorem}{Theorem}

\addbibresource{refs.bib}

\title{Avaliação e Estimação dos Parâmetros da Distribuição Skellam}
\author[1]{Antonino Alves Feitosa Neto}
\affil[1]{\href{mailto:antonino\_feitosa@yahoo.com.br} {antonino\_feitosa@yahoo.com.br}}
\affil[1]{Universidade Federal do Rio Grande do Norte}

\date{\today}

\begin{document}

\maketitle

\begin{abstract}
    A distribuição de Skellam é aplicada em diferentes áreas como esportes, saúde, finanças, etc. Assim, é relevante a investigação do comportamento de seus estimadores pontuais em cenários de tamanho amostral pequeno, ou seja, em cenários que o desempenho deles pode ser afetado pela presença de viés. Desse modo, investigamos o desempenho dos estimadores obtidos pelo método de máxima verossimilhança e pelo método dos momentos, comparando-os com as suas respectivas versões corrigidas por Bootstrap. Eles são comparados em termos do viés e do erro quadrático médio obtidos por meio de simulações de Monte Carlo. Verificamos que o estimador pelo método de máxima verossimilhança apresenta o menor erro quadrático médio em todos os caso, apesar de ter um viés maior que os demais. 
    
    \hfill
    
    \textbf{Palavras-chave:} \textit{Bootstrap}, Correção de viés, Máxima Verossimilhança, Método dos Momentos, Simulação de Monte Carlo
\end{abstract}

\section{Introdução}

A distribuição de Skellam foi proposta por Skellam \cite{skellam1946} para modelagem da diferença entre duas contagens. Ela é derivada da diferença entre duas variáveis aleatórias (V.A.) independentes que seguem uma distribuição de Poisson e por esse motivo ela também é chamada de distribuição de diferenças de Poisson.

Ela foi aplicada em diferentes áreas como esportes, finanças, saúde, etc. Por exemplo: Catania et al. \cite{catania2022} utiliza a distribuição de Skellam na modelagem da variação de preços de ações; Jiang et al. \cite{jiang2014} a aplica na identificação de padrões em expressão genética; já Karlis e Ntzoufras \cite{karlis2008} a utilizam na previsão do resultado de partidas de futebol.

Além dessas aplicações, a plataforma arXiv \cite{arXiv} apresenta diferentes trabalhos em processo de publicação baseados na distribuição de Skellam. Destacamos o trabalho de Rave e Kauermann \cite{rave2023} que modela o fluxo de entrada e saída diários de uma unidade de tratamento intensiva durante a pandemia de Covid-19 na Alemanha. Também citamos o trabalho de Wang \cite{wang2023} que aplica a distribuição no desenvolvimento de um algoritmo de aprendizado de máquina para sistemas de recomendação. Assim, percebemos a importância dessa distribuição e sua contribuição em diferentes áreas.

O objetivo deste trabalho é avaliar o desempenho de estimadores pontuais da distribuição de Skellam em cenários de tamanho amostral pequeno. Nesse contexto, os estimadores usuais podem apresentar um desempenho insatisfatório devido à presença de tendências quanto utilizamos pequenas amostras. Logo, desejamos avaliar os estimadores pontuais pelo método de máxima verossimilhança (EMV) e pelo método dos momentos (EMM), comparando-os com as suas respectivas versões corrigidas por Bootstrap. Assim, ao final do trabalho, desejamos identificar em quais cenários (tamanhos amostrais) eles são mais adequados.

O restante do trabalho se divide nas seguintes seções: seção de Referencial Teórico, apresentando uma breve revisão da distribuição de Skellam e dos métodos utilizados; seguida pela seção de Metodologia, que apresenta como os métodos serão aplicados para alcançar os objetivos citados; a seção dos Resultados obtidos, incluindo a discussão de cada um; terminando pela seção de considerações finais.


\section{Referencial Teórico}

Irwin \cite{irwin1937} apresenta a distribuição de probabilidade obtida da diferença de duas V.A. independentes e idênticas seguindo uma distribuição de Poisson. Skellam \cite{skellam1946} generaliza essa distribuição para o caso de parâmetros diferentes. Assim, a distribuição de Skellam é definida na Definição \ref{def:pmf} \cite{karlis2008}:

\begin{definition}[Distribuição de Skellam $PD(\theta_1, \theta_2)$]
    \label{def:pmf}
    Considere o par de variáveis aleatórias $(X,Y)$ tais que $X \sim Poisson(\theta_1)$ independente de $Y \sim Poisson(\theta_2)$. Então $Z=X-Y$ é dita ter uma distribuição de Skellam, denotada por PD($\theta_1, \theta_2$), e sua função massa de probabilidade é dada por:
    \begin{equation*}
      P(Z=z) = e^{-(\theta_1+\theta_2)} \left({\theta_1\over\theta_2}\right)^{x/2}I_{z}(2\sqrt{\theta_1\theta_2}) \qquad, z=...,-1,0,1,...
    \end{equation*}
    onde $I_y(x) = \left ( x \over 2 \right )^y { \sum_{k=0}^{\infty}  \left ( x^2 \over 4 \right )^k \over k! (y+k)! }$ é a função de Bessel modificada de primeiro tipo.
\end{definition}


Observe que o suporte dessa distribuição abrange os conjunto dos valores inteiros. Os valores negativos representam situações em que a V.A. $Y$ assume valores maiores que a $X$. Além disso, os parâmetros $\theta_1$ e $\theta_2$ assumem valores reais positivos, pois são parâmetros de uma distribuição de Poisson.

A partir da sua definição, como a diferença de duas V.A. independentes de Poisson, podemos obter a esperança e a variância de uma V.A. $Z$ com distribuição $PD(\theta_1, \theta2)$:

\begin{theorem}[Esperança de $PD(\theta_1, \theta_2)$]
    Considere a V.A $Z \sim PD(\theta_1, \theta2)$. O valor esperado $E[Z]$ e variância são definidos como $Var[Z]$:
    \begin{equation*}
        E[Z] = E[X - Y] = E[X] - E[Y] = \theta_1 - \theta_2
    \end{equation*}
    \begin{equation*}
        Var[Z] = Var[X-Y] = Var[X] + (-1)^2Var[Y] = \theta_1 + \theta_2,
    \end{equation*}
em que X e Y são duas V.A. independentes de uma distribuições de Poisson com parâmetro $\theta_1$ e $\theta_2$ respectivamente.
\end{theorem}

Ademais, $PD(\theta_1, \theta_2)$ possui as seguintes propriedades:

\begin{equation*}
    P(Z=z | \theta_1, \theta_2) = P(Z=-z | \theta_2, \theta_1).
\end{equation*}

Sua função geradora de momentos é definida como:

\begin{equation*}
    M_z(t) = exp\{ -(\theta_1 + \theta_2) + \theta_1 e^t  + \theta_1 e^{-t} \}.
\end{equation*}


E sua assimetria é definida como:

\begin{equation*}
    \beta_1 = {\theta_1 - \theta_2 \over ( \theta_1 + \theta_2)^{3 \over 2}}.
\end{equation*}

Desse modo, a distribuição apresenta uma assimetria positiva quando $\theta_1 > \theta_2$, negativa quando $\theta_1 < \theta_2$, e será simétrica quando $\theta_1 = \theta_2$. Para mais informações de como essas propriedades foram obtidas e também de propriedades adicionais, podem ser encontradas nos trabalhos de \cite{karlis2008,alzaid2010}.

Considerando os objetivos deste trabalho, também apresentaremos os estimadores obtidos pelo EMM e pelo EMV para os parâmetros $\theta_1$ e $\theta_2$. Além disso, as próximas seções também apresentam um breve resumo das principais técnicas utilizadas para avaliar a qualidade desses estimadores de forma prática.


\subsection{Estimadores de $\theta_1$ e $\theta_2$}

Ao ajustar um modelo, devemos estimar os valores dos parâmetros de modo mais fidedigno possível aos dados observados. Em geral, podemos dividir os métodos de estimação em pontual e intervalar, ou seja, estimar um único ponto ou um intervalo com certo nível de confiança para o valor real do parâmetro.

A estimação pontual é mais simples e fácil de interpretar, sendo valiosa quando se deseja comunicar informações a um público não especializado. No entanto, ela não reflete a incerteza associada à estimação, além de ser sensível a valores aberrantes. Esses problemas podem ser tratados por uma estimação intervalar. Logo, cada uma tem o seu mérito e em muitos casos apresentamos as duas estimativas dos dois tipos para fornecer uma visão ampla da inferência realizada.

Assim, investigaremos a qualidade dos EMM e do EMV que são estimadores pontuais. Esses métodos podem ser encontrados Bolfarine \cite{bolfarine2010} influindo descrições de suas propriedades. Para a distribuição de Skellam, os EMM são definidos na Definição \ref{def:emm} \cite{karlis2008}.

\begin{definition}[EMM]
    \label{def:emm}
    Seja $Z_1, Z_2, ..., Z_N$ uma amostra aleatória da V.A. $Z$ tal que $Z \sim PD(\theta_1, \theta_2)$. Então, os EMM de $\theta_1, \theta2$
    \begin{equation*}
    \tilde \theta_1 =  (S_z^2 + \bar{Z}) /2
    \end{equation*}
     \begin{equation*}
    \tilde \theta_2 =  (S_z^2 - \bar{Z}) /2,
    \end{equation*}
    em que $\bar{Z}$ é a média e $S_z^2$ a variância amostral de $Z_1, Z_2, ..., Z_N$, respectivamente. Os estimadores não são definidos caso $S_z^2 - |\bar{Z}| < 0$.
\end{definition}

Apesar da obtenção dos EMM serem analiticamente simples, não é uma tarefa trivial definir os EMV, pois envolve a resolução de um sistema de equações não lineares. Desse modo, podemos recorrer a métodos numéricos para obtenção dos EMV.

Uma vez que as estimativas foram obtidas, precisamos decidir qual delas é mais fidedigna ao valor real do parâmetro, ou seja, decidir entre as estimativas de EMM ou EMV. Podemos verificar na literatura que a medida de erro quadrático médio (EQM) pode ser utilizada como métrica de comparação da qualidade de estimadores. O erro quadrático médio de um estimador $\hat \theta$ de um parâmetro $\theta$ pode ser definido em termos do viés $Vi\acute es(\hat \theta)$ e variância $Var(\hat \theta)$ desse estimador como na Equação \ref{eq:eqm} \cite{bolfarine2010}:

\begin{equation*}
    \label{eq:eqm}
    EQM(\hat \theta) = Var(\hat \theta) + Vi\acute es(\hat \theta).
\end{equation*}

Porém, calcular o viés e a variância de um estimador de modo analítico pode ser uma tarefa complexa, pois pode resultar na resolução de equações não lineares. Novamente, podemos recorrer a métodos numéricos para a determinação desses valores, em particular, podemos recorrer ao método estatístico de Monte Carlo para determinação da esperança e da variância de um estimador, e por consequência, a determinação do erro quadrático médio. 


\subsection{Método de Monte Carlo}

Esta seção descreve brevemente como podemos utilizar o método de Monte Carlo para obter a esperança e a variância de um estimador. Uma descrição detalhada desse método, incluindo teoria e exemplos, pode ser encontrada em \cite{robert2000}.

Desse modo, considere o problema de estimar o valor esperado de uma V.A. $X$ qualquer e suponha que possam ser geradas amostras aleatórias com a mesma distribuição de probabilidade de $X$. Chamamos a geração de um valor de simulação e caso sejam geradas $r$ simulações, isto é, $X_1, X_2, ..., X_r$, podemos calcular a média amostral $\bar{X}$. Logo, como consequência da lei dos grandes números \cite{ross2009}, $\bar{X}$ converge para $E[X]$ quando $r \to \infty$.

Portanto, se desejamos estimar a esperança de um estimador $E[\hat \theta]$, precisamos de um método de geração de variáveis aleatórias com a mesma distribuição de probabilidade de $\hat \theta$ e de um número de réplicas $r$ grande de amostras. E assim podemos obter uma estimativa numérica de $E[\hat \theta]$ e também de $Var[\hat \theta]$, ou seja, temos como calcular o EQM de um estimador $\hat \theta$.


\subsection{Método de Bootstrap}

Os métodos de Bootstrap foram introduzidos por Efrom \cite{efron1979}. Correspondem a uma classe de métodos de Monte Carlo para aproximação de uma distribuição de probabilidade por meio de uma função empírica obtida de uma amostra finita.

O termo \textit{Bootstrap} é oriundo da expressão da língua inglesa “\textit{to pull oneself up by one’s bootstrap}” que passa a ideia de alcançar sucesso por esforço próprio, iniciando de circunstâncias muito difíceis e sem ajuda, como emergir de um afogamento puxando pela alça do próprio sapato \cite{collins2023}. Na estatística, pode ser interpretado como a capacidade de obter propriedades de uma população a partir de poucas observações.

Desse modo, considere que desejamos estimar um parâmetro de interesse $\theta$, que temos uma amostra de obervações $x_1, x_2, ..., x_n$ e que dispomos de um estimador para $\theta$ de forma $\hat \theta = f(x_1, x_2, ..., x_n)$. Assim, a estimativa ${\hat \theta}^*$ para $\theta$ é obtida pela média amostral $\bar{{\hat \theta}^*}$ dos estimadores de $B$ amostras bootstrap. Uma amostra boostrap é uma amostra com reposição obtida de $x_1, x_2, ..., x_n$ (de mesmo tamanho que a observada), assumindo que cada $x_i$ possui a mesma probabilidade de ocorrência $1/n$, usada para obter uma estimativa ${\hat \theta}^{(b)}$. A distribuição de $\bar{{\hat \theta}^*}$ converge para a distribuição de $\theta$ quando $B$ tende ao infinito \cite{efron1993}.

Observe que não fazemos suposições sobre a distribuição de probabilidade da amostra de observações, ou seja, é um método não paramétrico. No entanto, caso a distribuição seja conhecida, as amostras bootstrap podem ser substituídas por amostras aleatórias e independentes da distribuição, ou seja, temos um método de Monte Carlo. Nesse último caso, classificamos o método Bootstrap como paramétrico.

Apesar de ${\hat \theta}^*$ ser uma estimativa de $\theta$, geralmente ele não aproxima bem a locação da distribuição e, portanto, não dever ser usado como estimador para $\theta$. Dentre as várias aplicações, podemos usar o método de Bootstrap para avaliação e correção do viés de um estimador \cite{efron1993}.

Assim, considere o problema de remover o viés de um estimador $\hat \theta$, ou seja, desejamos obter $\hat \theta ^ c = \hat \theta - Vi\acute es(\hat \theta)$ tal que $\theta = \hat \theta ^ c$. Porém, não temos como calcular $Vi\acute es(\hat \theta) = E[\hat \theta] - \theta$, pois $\theta$ é desconhecido. No entanto, podemos utilizar a estimativa de Boostrap para $\hat \theta$ como mostrado na Equação \ref{eq:cbias}\cite{efron1993}.

\begin{equation*}
    \label{eq:cbias}
    \hat \theta ^ c = \hat \theta - Vi\acute es({\hat \theta}^*) =\hat \theta - (E[{\hat \theta}^*] - \hat \theta) = 2 \hat \theta - \bar{{\hat \theta}^*}.
\end{equation*}

Ademais, também efetuaremos a correção de viés por Bootstrap, de modo paramétrico, dos estimadores de EMM e EMV, ou seja, serão comparados quatro métodos.

\section{Metodologia}

Consideraremos quatro cenários para a avaliação dos estimadores em função dos valores reais dos parâmetros, no seguinte conjunto $PD(\theta_1=10,\theta_2=10)$, $PD(\theta_1=10,\theta_2=100)$, $PD(\theta_1=100,\theta_2=10)$ e $PD(\theta_1=100,\theta_2=100)$. Escolhemos esses valores baseados em aplicações de esportes, por exemplo, cada equipe marca cerca de 1 a 2 pontos em jogos de futebol, já para o handebol são cerca de 20 pontos, enquanto no basquete são cerca de 100 pontos. Os resultados das partidas desses esportes podem ser modelados por distribuições de Skellam e por isso escolhemos os valores como combinações de 10 e 100.

Para cada par de valores dos parâmetros, o EQM de cada estimador será aproximado pelo método de Monte Carlo gerando 2000 amostras com os valores reais dos parâmetros. Isto é, obtermos o EQM dos estimadores de EMM ($\hat \theta_1$), EMV e de suas versões com viés corrigido por Bootstrap paramétrico, usando 500 amostras bootstrap. Essas quantidades foram escolhidas considerando o tempo e recursos computacionais para a realização do trabalho.

Além disso, cada cenário será executado com tamanhos amostrais $n$ de 25, 50, 75 e 100. Esses valores foram escolhidos por representarem amostras pequenas e que provavelmente apresentarão um alto viés. Por fim, os resultados serão comparados em termos da magnitude do EQM, variância e viés de cada estimador.

A implementação foi efetuada no R \cite{R2023} com o auxílio do pacote Skellam \cite{SKELLAM106} para geração de valores aleatórios e função de probabilidade da distribuição. O software está disponível no repositório Skellam-Distribution \cite{antonino2023}.

\section{Resultados}

A Tabela \ref{tab:10x10} apresenta as métricas dos estimadores para a distribuição de probabilidades $PD(\theta_1=10, \theta_2=10)$. Observa-se que o EMV $\hat{\theta}$ apresenta o menor EQM para todos os tamanhos amostrais, porém, com o maior valor de viés. Apesar de sua versão corrigida apresentar um menor viés em todos os tamanhos amostrais, ele gera um amento da variância e um maior EQM.

Nota-se que as diferenças de magnitudes entre as métricas dos $\tilde{\theta}$ e dos $\tilde{\theta}^*$ são praticamente desprezíveis, menores que 10 vezes em relação à magnitude do EQM. Isso sugere que os EMM para a distribuição de Skellam não são viesados nesse cenário. Além disso, com o aumento do tamanho amostral, verificamos uma diminuição do EQM em todos os estimadores.

Desse modo, conclui-se que devemos utilizar o EMV $\hat{\theta}$ para a estimação de parâmetros quando a distribuição real é $PD(\theta_1=10, \theta_2=10)$ e que não há vantagens em aplicar correção de viés nesse caso. Os resultados da Tabela \ref{tab:100x100} são similares aos da Tabela \ref{tab:10x10}, com os dados escalados por um fator de 100. Logo, obtêm-se as mesmas conclusões para a distribuição real $PD(\theta_1=100, \theta_2=100)$.


\begin{table}[htbp]
\centering
\begin{tabular}{c|c|rrrr|rrrr}
 \multicolumn{2}{c}{} & \multicolumn{4}{|c|}{Estimativas de $\theta_1 = 10$} &  \multicolumn{4}{c}{Estimativas de $\theta_2 = 10$} \\
  \hline
 $n$ & Est. & Média & Viés & Var & EQM & Média & Viés & Var & EQM \\ 
  \hline
  
  25 & $\hat{\theta}$ & 9.69 & $-$0.31 & 8.19 & \textbf{8.29} & 9.69 & $-$0.31 & 8.34 & \textbf{8.44} \\ 
     & $\hat{\theta}^*$ & 10.05 & 0.05 & 8.88 & 8.88 & 10.05 & 0.05 & 9.04 & 9.04 \\ 
     & $\tilde{\theta}$ & 10.07 & 0.07 & 8.86 & 8.87 & 10.07 & 0.07 & 9.02 & 9.02 \\ 
     & $\tilde{\theta}^*$ & 10.07 & 0.07 & 8.85 & 8.85 & 10.06 & 0.06 & 9.00 & 9.00 \\ 
   \hline
   
   50 & $\hat{\theta}$ & 9.79 & $-$0.21 & 4.35 & \textbf{4.39} & 9.78 & $-$0.22 & 4.26 & \textbf{4.31} \\ 
      & $\hat{\theta}^*$ & 9.98 & $-$0.02 & 4.52 & 4.53 & 9.96 & $-$0.04 & 4.44 & 4.44 \\ 
      & $\tilde{\theta}$ & 9.98 & $-$0.02 & 4.52 & 4.53 & 9.96 & $-$0.04 & 4.44 & 4.44 \\ 
      & $\tilde{\theta}^*$ & 9.98 & $-$0.02 & 4.55 & 4.55 & 9.96 & $-$0.04 & 4.45 & 4.45 \\

    \hline
    75 & $\hat{\theta}$ & 9.86 & $-$0.14 & 2.84 & \textbf{2.86} & 9.86 & $-$0.14 & 2.87 & \textbf{2.89} \\ 
        & $\hat{\theta}^*$ & 9.98 & $-$0.02 & 2.90 & 2.90 & 9.98 & $-$0.02 & 2.93 & 2.94 \\ 
        & $\tilde{\theta}$ & 9.98 & $-$0.02 & 2.92 & 2.92 & 9.98 & $-$0.02 & 2.96 & 2.96 \\ 
        & $\tilde{\theta}^*$ & 9.98 & $-$0.02 & 2.93 & 2.93 & 9.98 & $-$0.02 & 2.96 & 2.96 \\ 

    \hline
    100 & $\hat{\theta}$ & 9.92 & $-$0.08 & 2.08 & \textbf{2.09} & 9.91 & $-$0.09 & 2.10 & \textbf{2.11} \\ 
        & $\hat{\theta}^*$ & 10.02 & 0.02 & 2.13 & 2.13 & 10.00 & 0.00 & 2.15 & 2.15 \\ 
        & $\tilde{\theta}$ & 10.02 & 0.02 & 2.13 & 2.13 & 10.00 & $-$0.00 & 2.15 & 2.15 \\ 
        & $\tilde{\theta}^*$ & 10.01 & 0.01 & 2.14 & 2.14 & 10.00 & $-$0.00 & 2.15 & 2.15 \\ 
    \hline
\end{tabular}
\caption{Métricas dos estimadores para $\theta_1 = 10$ e $\theta_2 = 10$.}
\label{tab:10x10}
\end{table}



\begin{table}[htbp]
\centering
\begin{tabular}{c|c|rrrr|rrrr}
 \multicolumn{2}{c}{} & \multicolumn{4}{|c|}{Estimativas de $\theta_1=100$} &  \multicolumn{4}{c}{Estimativas de $\theta_2=100$} \\
  \hline
 $n$ & Est. & Média & Viés & Var & EQM & Média & Viés & Var & EQM \\ 
  \hline
    25 & $\hat{\theta}$ & 96.02 & $-$3.98 & 804.51 & \textbf{820.35} & 95.98 & $-$4.02 & 806.14 & \textbf{822.27} \\ 
        & $\hat{\theta}^*$ & 99.81 & $-$0.19 & 870.50 & 870.54 & 99.77 & $-$0.23 & 872.19 & 872.24 \\ 
        & $\tilde{\theta}$ & 99.99 & $-$0.01 & 872.92 & 872.92 & 99.96 & $-$0.04 & 874.58 & 874.59 \\ 
        & $\tilde{\theta}^*$ & 100.01 & 0.01 & 878.74 & 878.74 & 99.97 & $-$0.03 & 880.13 & 880.13 \\ 
  
   \hline
    50 & $\hat{\theta}$ & 98.14 & $-$1.86 & 397.35 & \textbf{400.82} & 98.11 & $-$1.89 & 397.59 & \textbf{401.17} \\ 
        & $\hat{\theta}^*$ & 100.06 & 0.06 & 413.97 & 413.98 & 100.04 & 0.04 & 413.98 & 413.98 \\ 
        & $\tilde{\theta}$ & 100.13 & 0.13 & 413.60 & 413.62 & 100.10 & 0.10 & 413.85 & 413.86 \\ 
        & $\tilde{\theta}^*$ & 100.14 & 0.14 & 414.67 & 414.69 & 100.11 & 0.11 & 414.87 & 414.88 \\

    \hline
    75 & $\hat{\theta}$ & 98.47 & $-$1.53 & 253.60 & \textbf{255.93} & 98.50 & $-$1.50 & 254.04 & \textbf{256.28} \\ 
        & $\hat{\theta}^*$ & 99.78 & $-$0.22 & 260.74 & 260.79 & 99.81 & $-$0.19 & 261.23 & 261.27 \\ 
        & $\tilde{\theta}$ & 99.79 & $-$0.21 & 260.46 & 260.50 & 99.82 & $-$0.18 & 260.90 & 260.94 \\ 
        & $\tilde{\theta}^*$ & 99.79 & $-$0.21 & 260.74 & 260.79 & 99.82 & $-$0.18 & 261.29 & 261.33 \\ 

    \hline
    100 & $\hat{\theta}$ & 98.61 & $-$1.39 & 197.15 & \textbf{199.09} & 98.54 & $-$1.46 & 197.16 & \textbf{199.29} \\ 
        & $\hat{\theta}^*$ & 99.56 & $-$0.44 & 201.36 & 201.55 & 99.50 & $-$0.50 & 201.26 & 201.51 \\ 
        & $\tilde{\theta}$ & 99.59 & $-$0.41 & 201.06 & 201.23 & 99.53 & $-$0.47 & 201.07 & 201.29 \\ 
        & $\tilde{\theta}^*$ & 99.61 & $-$0.39 & 201.64 & 201.79 & 99.54 & $-$0.46 & 201.58 & 201.79 \\
    \hline
\end{tabular}
\caption{Métricas dos estimadores para $\theta_1 = 100$ e $\theta_2 = 100$.}
\label{tab:100x100}
\end{table}

Porém, para valores distintos dos parâmetros, em que os resultados são apresentados nas Tabelas \ref{tab:10x100} e \ref{tab:100x10}, nota-se um comportamento diferente. O EMV $\hat{\theta}$ ainda apresenta o menor EQM e também verifica-se uma diminuição do EQM com o aumento do tamanho amostral. No entanto, os estimadores corrigidos $\hat{\theta}^*$ e $\tilde{\theta}^*$ apresentam maior viés e EQM que suas respectivas versões $\hat{\theta}$ e $\tilde{\theta}$ em quase todos os casos. Isso é mais evidente para o parâmetro assumindo o menor valor, $\theta_1 = 10$ em \ref{tab:10x100} e $\theta_2 = 10$ em \ref{tab:100x10}.


\begin{table}[htbp]
\centering
\begin{tabular}{c|c|rrrr|rrrr}
 \multicolumn{2}{c}{} & \multicolumn{4}{|c|}{Estimativas de $\theta_1=10$} &  \multicolumn{4}{c}{Estimativas de $\theta_2=100$} \\
  \hline
 $n$ & Estimador & Média & Viés & Var & EQM & Média & Viés & Var & EQM \\ 
  \hline
   25 & $\hat{\theta}$ & 10.24 & 0.24 & 150.68 & \textbf{150.74} & 100.18 & 0.18 & 153.41 & \textbf{153.44} \\ 
    & $\hat{\theta}^*$ & 8.90 & $-$1.10 & 216.30 & 217.51 & 98.84 & $-$1.16 & 218.42 & 219.76 \\ 
    & $\tilde{\theta}$ & 11.96 & 1.96 & 178.46 & 182.30 & 99.66 & $-$0.34 & 257.08 & 257.19 \\ 
    & $\tilde{\theta}^*$ & 9.28 & $-$0.72 & 224.19 & 224.71 & 99.66 & $-$0.34 & 258.18 & 258.30 \\ 
  
   \hline
    50 & $\hat{\theta}$ & 9.71 & $-$0.29 & 78.53 & \textbf{78.62} & 99.74 & $-$0.26 & 80.64 & \textbf{80.70} \\ 
        & $\hat{\theta}^*$ & 9.00 & $-$1.00 & 107.70 & 108.70 & 99.03 & $-$0.97 & 109.56 & 110.51 \\ 
        & $\tilde{\theta}$ & 10.64 & 0.64 & 86.24 & 86.65 & 99.79 & $-$0.21 & 113.24 & 113.29 \\ 
        & $\tilde{\theta}^*$ & 9.18 & $-$0.82 & 108.84 & 109.52 & 99.79 & $-$0.21 & 113.79 & 113.83 \\  

    \hline
    75 & $\hat{\theta}$ & 9.72 & $-$0.28 & 64.64 & \textbf{64.71} & 99.75 & $-$0.25 & 66.26 & \textbf{66.33} \\ 
        & $\hat{\theta}^*$ & 9.29 & $-$0.71 & 83.93 & 84.44 & 99.32 & $-$0.68 & 85.57 & 86.03 \\ 
        & $\tilde{\theta}$ & 10.37 & 0.37 & 69.57 & 69.71 & 99.90 & $-$0.10 & 84.34 & 84.35 \\ 
        & $\tilde{\theta}^*$ & 9.37 & $-$0.63 & 84.87 & 85.26 & 99.89 & $-$0.11 & 84.41 & 84.42 \\

    \hline
    100 & $\hat{\theta}$ & 9.61 & $-$0.39 & 51.82 & \textbf{51.97} & 99.61 & $-$0.39 & 52.28 & \textbf{52.43} \\ 
        & $\hat{\theta}^*$ & 9.31 & $-$0.69 & 64.62 & 65.09 & 99.31 & $-$0.69 & 64.98 & 65.45 \\ 
        & $\tilde{\theta}$ & 10.16 & 0.16 & 55.24 & 55.26 & 99.86 & $-$0.14 & 63.07 & 63.09 \\ 
        & $\tilde{\theta}^*$ & 9.47 & $-$0.53 & 66.21 & 66.49 & 99.88 & $-$0.12 & 63.34 & 63.35 \\
    \hline
\end{tabular}
\caption{Métricas do estimadores para $\theta_1 = 10$ e $\theta_2 = 100$.}
\label{tab:10x100}
\end{table}


\begin{table}[htbp]
\centering
\begin{tabular}{c|c|rrrr|rrrr}
 \multicolumn{2}{c}{} & \multicolumn{4}{|c|}{Estimativas de $\theta_1=100$} &  \multicolumn{4}{c}{Estimativas de $\theta_2=10$} \\
  \hline
 $n$ & Estimador & Média & Viés & Var & EQM & Média & Viés & Var & EQM \\ 
  \hline
    25 & $\hat{\theta}$ & 100.61 & 0.61 & 147.29 & \textbf{147.66} & 10.62 & 0.62 & 143.63 & \textbf{144.01} \\ 
        & $\hat{\theta}^*$ & 99.38 & $-$0.62 & 211.99 & 212.37 & 9.39 & $-$0.61 & 208.90 & 209.28 \\ 
        & $\tilde{\theta}$ & 100.45 & 0.45 & 244.03 & 244.23 & 12.46 & 2.46 & 171.45 & 177.51 \\ 
        & $\tilde{\theta}^*$ & 100.45 & 0.45 & 244.86 & 245.06 & 9.87 & $-$0.13 & 216.92 & 216.94 \\ 
  
   \hline
    50 & $\hat{\theta}$ & 99.62 & $-$0.38 & 89.31 & \textbf{89.45} & 9.60 & $-$0.40 & 86.70 & \textbf{86.86} \\ 
        & $\hat{\theta}^*$ & 98.85 & $-$1.15 & 120.01 & 121.33 & 8.82 & $-$1.18 & 117.49 & 118.87 \\ 
        & $\tilde{\theta}$ & 99.56 & $-$0.44 & 126.80 & 126.99 & 10.58 & 0.58 & 94.94 & 95.28 \\ 
        & $\tilde{\theta}^*$ & 99.58 & $-$0.42 & 127.24 & 127.41 & 9.08 & $-$0.92 & 118.52 & 119.36 \\ 

    \hline
    75 & $\hat{\theta}$ & 99.82 & $-$0.18 & 66.13 & \textbf{66.17} & 9.84 & $-$0.16 & 65.33 & \textbf{65.36} \\ 
        & $\hat{\theta}^*$ & 99.38 & $-$0.62 & 85.18 & 85.57 & 9.40 & $-$0.60 & 84.54 & 84.90 \\ 
        & $\tilde{\theta}$ & 99.97 & $-$0.03 & 84.38 & 84.38 & 10.51 & 0.51 & 69.75 & 70.01 \\ 
        & $\tilde{\theta}^*$ & 99.98 & $-$0.02 & 84.68 & 84.68 & 9.54 & $-$0.46 & 85.42 & 85.63 \\ 

    \hline
    100 & $\hat{\theta}$ & 99.71 & $-$0.29 & 52.30 & \textbf{52.39} & 9.72 & $-$0.28 & 51.27 & \textbf{51.35} \\ 
        & $\hat{\theta}^*$ & 99.45 & $-$0.55 & 65.15 & 65.45 & 9.46 & $-$0.54 & 64.16 & 64.45 \\ 
        & $\tilde{\theta}$ & 99.97 & $-$0.03 & 62.25 & 62.25 & 10.25 & 0.25 & 54.42 & 54.49 \\ 
        & $\tilde{\theta}^*$ & 99.98 & $-$0.02 & 62.19 & 62.20 & 9.57 & $-$0.43 & 64.77 & 64.96 \\ 
    \hline
\end{tabular}
\caption{Métricas do estimadores para $\theta_1 = 100$ e $\theta_2 = 10$.}
\label{tab:100x10}
\end{table}


Desse modo, conclui-se que a correção de viés apresenta pouca ou quase nenhuma melhora em relação aos estimadores não corrigidos e que o EMV $\hat{\theta}$ apresenta os menores valores de EQM em todos os tamanhos amostrais. 

\section{Considerações Finais}

Este trabalho avaliou o desempenho dos estimadores pontuais pelo método de máxima verossimilhança e pelo método dos momentos para a distribuição de Skellam em cenários de tamanho amostral pequeno. Além disso, os estimadores foram comparados com suas versões corrigidas por Bootstrap para os tamanhos amostrais de 25, 50, 75 e 100, com combinações dos valores de 10 e 100 para os parâmetros das distribuições reais.

Observou-se que o estimador de máxima verossimilhança apresentou o menor erro quadrático médio em todos os cenários e que não houve melhorias no EQM ao corrigi-lo por Bootstrap. Além disso, verificou-se um aumento do viés dos estimadores corrigidos em relação as suas respectivas versões não corrigidas ao empregar valores diferentes dos parâmetros das distribuições reais, ou seja, a correção não foi efetiva nesses cenários.

Assim, recomendamos o uso do estimador de máxima verossimilhança para tamanhos amostrais pequenos sem a necessidade de correção de viés. No entanto, seria interessante verificar esses resultados para amostras ainda menores, dependendo da necessidade da aplicação.


\medskip

\printbibliography
\end{document}









